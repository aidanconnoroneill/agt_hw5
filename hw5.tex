\documentclass[11pt]{article}
\usepackage{fullpage}
\usepackage{clrscode3e}
\usepackage[fleqn]{amsmath}
\usepackage{amsthm,amssymb}
\usepackage{color}
\usepackage[shortlabels]{enumitem}
\usepackage{multicol}
\usepackage{hyperref}
\usepackage{mathtools}


\setlength{\parskip}{2mm}
\setlength{\parindent}{0mm}

\newcommand{\titlebox}[4]{
    \begin{center}
        \framebox{
            \vbox{
            \hbox to \textwidth { #1 \hfill #3}
            \vspace{-4mm}
            \hbox to \textwidth {\hfill \Large \bf #2 \hfill}
            \hbox to \textwidth { {\it \hfill Due #4 \hfill} }
        }
    }
    \end{center}
}


\newcommand{\matr}[1]{\mathbf{#1}}


\begin{document}

\titlebox{CSC 383}
{Homework 5}
{Zach Nussbaum and Aidan O'Neill} % replace this line with: {your names} (and remove \textcolor command)
{April 3\textsuperscript{rd}, 2020}



%%%%%%%%%%%%%%%%%%%%%%%%%%%%%%%%%%%%%%%%%%%%%%%%%%%%%%%%%%%%%%%%
\subsection*{Problem 1}

A symmetric game is one where any permutation to the order of the players (e.g. swapping players 1 and 2) does not change the game.
For such permutations to have no effect, it must be the case that all players have identical action sets, and that the payoff to any outcome depends only on \emph{how many} players select each action, not which specific players chose an action.

A corollary of Nash's theorem that appears in Nash's original 1951 paper is that every symmetric game has a symmetric Nash equilibrium.
In a symmetric game, a mixed-strategy profile is symmetric if all players are following the same mixed strategy.
To prove this corollary, we will take a different approach, based on the symmetric replicator dynamics function.

When running replicator dynamics on a symmetric game to search for a symmetric equilibrium, we don't need to keep track of separate strategies for each player.
Instead, we can use a single mixed strategy, that will be played by all players, to represent the profile.
We can then treat the update done on each round of replicator dynamics as a function that maps a mixed strategy to a mixed strategy.
We will use this function to prove the existence of symmetric equilibria in symmetric games.

\begin{enumerate}[(a)]

\item
Define as a mathematical function the update performed by one iteration of replicator dynamics.

\fbox{
\begin{minipage}{\textwidth}

$$ \sigma '(a^1) = \sigma(a_i)\frac{[\mathbb{E}u(a_1, \sigma_{-i})-offset]}{\sum_{a_j\in A}\sigma(a_j)[\mathbb{E}u(a_j, \sigma_{-i})-offset]} $$
We note that this is a symmetric game, so we need not iterate over different players, as the utilities are the same for each player.  We combine the update per weight and the normalization into one step, and represent the \texttt{DevPay} function with expected value notation.  
\end{minipage}
}

\item
Prove that a mixed strategy is a symmetric Nash equilibrium if and only if it is a fixed point of this function.

\fbox{
\begin{minipage}{\textwidth}
We decompose the $\iff$ into two parts: the $\Rightarrow$ and $\Leftarrow$.  \\
$\Rightarrow$: \textbf{Claim:} If a mixed strategy is a symmetric Nash equilibrium, it is a fixed point of this function.\\ \textbf{Proof:} 
Consider the update performed by the replicator dynamics function on a mixed strategy Nash equilibrium, $\sigma(a_i)$.
\begin{align*}
\sigma '(a^1) &= \sigma(a_i)\frac{[\mathbb{E}u(a_1, \sigma_{-i})-offset]}{\sum_{a_j\in A}\sigma(a_j)[\mathbb{E}u(a_j, \sigma_{-i})-offset]}\\
\shortintertext{As this is a Nash Equilibrium, we know that $\mathbb{E}u(a_1, \sigma_{-i})$ is equal for all strategies that we mix between.  Factoring, we get}
\sigma '(a^1) &= \sigma(a_i)\frac{[\mathbb{E}u(a_1, \sigma_{-i})-offset]}{[\mathbb{E}u(a_j, \sigma_{-i})-offset]\sum_{a_j\in A}\sigma(a_j)}
\shortintertext{As $\sum_{a_j\in A}\sigma(a_j)$ is 1}
\sigma '(a^1) &= \sigma(a_i)\frac{[\mathbb{E}u(a_1, \sigma_{-i})-offset]}{[\mathbb{E}u(a_j, \sigma_{-i})-offset]}\\
\sigma '(a^1) &= \sigma(a_i)
\end{align*}
As the update performed by the replicator dynamic function does not change the mixed strategy profile, we have a fixed point.  \\
$\Leftarrow$: \textbf{Claim:} If the replicator dynamics function results does not change the mixed strategy profile, the mixed strategy profile is a Nash equilibrium.  \\ \textbf{Proof:} 
If the replicator dynamics function does not change the mixed strategy profile, and because we only consider those actions which we mix between with positive utility, we have that
\begin{align*}
1 &= \frac{[\mathbb{E}u(a_1, \sigma_{-i})-offset]}{\sum_{a_j\in A}\sigma(a_j)[\mathbb{E}u(a_j, \sigma_{-i})-offset]}\\
{\sum_{a_j\in A}\sigma(a_j)[\mathbb{E}u(a_j, \sigma_{-i})-offset]} &= [\mathbb{E}u(a_1, \sigma_{-i})-offset]
\end{align*}
We proceed by contradiction.  If this is not a Nash equilibrium, we know that the payoff one player would receive by deviating to one action is distinct from the payoff that same player would receive by deviating to another action; if they are indifferent between all actions they mix between, it is necessarily a Nash equilibrium.  Thus, there is some payoff deviation, $\mathbb{E}u(a_l, \sigma_{-i})$, which is smallest.  As, by our assumption, we know that there is at least one deviation payoff in the sum greater than $\mathbb{E}u(a_l, \sigma_{-i})$ and all of the terms are at least $\mathbb{E}u(a_l, \sigma_{-i})$, we know that our equality above does not hold, a contradiction.  Thus, if the replicator dynamics function does not change the mixed strategy profile, the mixed strategy profile is a Nash equilibrium.  \boxed{}
\end{minipage}
}

\item
Use Brouwer's fixed-point theorem to show that this function must have at least one fixed point.

\fbox{
\begin{minipage}{\textwidth}
For Brouwer's fixed-point theorem to apply to a function $f: S\rightarrow S$, we need for f to be continuous that maps $S$, a compact, convex set to itself.  We define $f$ as $f: \sigma(a_i)\frac{[\mathbb{E}u(a_1, \sigma_{-i})-offset]}{\sum_{a_j\in A}\sigma(a_j)[\mathbb{E}u(a_j, \sigma_{-i})-offset]} $
In addition, $S$ is the simplotope of all mixed-strategy profiles in the game.  
\begin{enumerate}
    \item $S$ is bounded as all coordinates lie within the hypercube with dimension $\mathbb{R}^{|A|}$.  
    \item $S$ is closed because 0 and 1 are valid probabilities so the set contains its boundary.  
    \item Consider two strategy profiles $\sigma_1$ and $\sigma_2$.  We show that their convex combination, $\sigma'$ is another valid strategy profile.  
        $\sigma' = u\sigma_1 + v\sigma_2$ for some $u$ and $v$ such that $u +v=1$ and $u, v\geq 0$.  Thus, 
        \begin{align*}
        \sigma' &= u\sigma_1 + (1-u)\sigma_2\\
        \sigma' &= u\sigma_1 + sigma_2 - u\sigma_2
        \end{align*}
        $\sigma'$ is clearly still a vector in $\mathbb{R}^{|A|}$.  As every element in $\sigma_1$ and $\sigma_2$ is nonnegative and $u$ and $v$ are nonnegative, we know that every element in $\sigma'$ is nonnegative.  Finally, $\sum_{a\in A} \sigma_1 = \sum_{a\in A} \sigma_2 = 1$, so $\sum_{a\in A}sigma' = u\cdot 1 + 1 - u\cdot 1 = 1$.  It follows that the convex combination of valid strategy profiles is a strategy profile, and $S$ is therefore a convex set.   
    \item Consider our function $f$.  We note that utility functions are continuous so $\mathbb{E}u(a_1, \sigma_{-i})$ is continuous.  In addition, offsets are constant and subtracting by an offset preserves continuity.  Finally, dividing by a continuous function by something preserves continuity so long as the denominator is not 0, which, because the sum of the action probabilities in a mixed strategy profile is always 1, our denominator never is.  Thus, we have that $f$ is continuous. 
\end{enumerate}
Thus, we have that the function the update performed by one iteration of replicator dynamics meets the criteria for the Brouwer's fixed-point theorem, so there must be a fixed point; this fixed point corresponds to a Nash equilibrium.  \boxed{}
\end{minipage}
}

\end{enumerate}






%%%%%%%%%%%%%%%%%%%%%%%%%%%%%%%%%%%%%%%%%%%%%%%%%%%%%%%%%%%%%%%%
\subsection*{Problem 2}

In the Jupyter notebook \texttt{nash\_learning.ipynb} implement the following functions:

\begin{itemize}
\item \texttt{fictitious\_play}
\item \texttt{replicator\_dynamics}
\item \texttt{random\_profile}
\item \texttt{RD\_random\_restarts}
\end{itemize}

Note that the version of replicator dynamics your are implementing is the asymmetric version, and is therefore not identical to the function you used in problem 1.
A function for plotting the traces of replicator dynamics and fictitious play will be available soon.


\end{document}